% $Header: /cvsroot/latex-beamer/latex-beamer/solutions/conference-talks/conference-ornate-20min.en.tex,v 1.7 2007/01/28 20:48:23 tantau Exp $

\documentclass[10pt]{beamer}

\mode<beamer>
{
  \usetheme{default}
  \usecolortheme[rgb={0,0,0.8}]{structure}
  %\setbeamercolor{normal text}{bg=blue!50}
  %\setbeamercolor{normal text}{fg=blue!50}
  % or ...

  %\setbeamercovered{transparent}
  % or whatever (possibly just delete it)
}


\usepackage[english]{babel}
% or whatever

%\usepackage[latin1]{inputenc}
% or whatever

\usepackage{times}
\usepackage[T1]{fontenc}
% Or whatever. Note that the encoding and the font should match. If T1
% does not look nice, try deleting the line with the fontenc.

%\usepackage{newcent}
%\usefonttheme{structuresmallcapsserif}

\usepackage{amssymb,latexsym,amsmath}
\usepackage{amsthm}
\DeclareMathOperator*{\argmin}{arg\,min}
\DeclareMathOperator*{\argmax}{arg\,max}

\usepackage{mathtools}
\input xy 
\xyoption{all}
\usepackage[latin1]{inputenc}
\usepackage{color}
\usepackage{tikz}


\title[Gaussian Process for Time Series Analysis] % (optional, use only with long paper titles)
{Gaussian Process for Time Series Analysis}

%\subtitle

\author[Dr. Juan Orduz] % (optional, use only with lots of authors)
{Dr. Juan Orduz}
% - Give the names in the same order as the appear in the paper.
% - Use the \inst{?} command only if the authors have different
%   affiliation.

\institute[PyData Berlin 2018] % (optional, but mostly needed)
{

}
% - Use the \inst command only if there are several affiliations.
% - Keep it simple, no one is interested in your street address.

\date[ PyData Berlin 2018] % (optional, should be abbreviation of conference name)
{ PyData Berlin 2019}
% - Either use conference name or its abbreviation.
% - Not really informative to the audience, more for people (including
%   yourself) who are reading the slides online

\subject{data science}
% This is only inserted into the PDF information catalog. Can be left
% out.



% If you have a file called "university-logo-filename.xxx", where xxx
% is a graphic format that can be processed by latex or pdflatex,
% resp., then you can add a logo as follows:

\pgfdeclareimage[height=0.7cm]{university-logo}{images/logo.png}
\logo{\pgfuseimage{university-logo}}

% If you wish to uncover everything in a step-wise fashion, uncomment
% the following command:

%\beamerdefaultoverlayspecification{<+->}


\begin{document}

\begin{frame}
  \titlepage
\end{frame}

%\begin{frame}{Contenido}
%\tableofcontents
%\end{frame}

\begin{frame}{Overview}
\tableofcontents
\end{frame}

\section{Introduction}

\begin{frame}{Multivariate Normal Distribution}
$X = (X_1, \cdots, X_d)$ has a{ \bf multivariate normal distribution} if every linear combination is normally distributed. In this case it has density of the form

$$
p(x|m,K_0) =\frac{1}{\sqrt{(2\pi)^{d}|K_0|}}\exp\left(-\frac{1}{2}(x - m)^\dagger K_0^{-1}(x - m)\right)
$$
%(2\pi)^{−d/2}|K_0|^{−1/2}\exp\left(-\frac{1}{2}(x−m)^T {K_0}^{-1}(x−m)\right)

where $m \in \mathbb{R}^d$ is the {\bf mean vector} and  $K_0 \in M_d(\mathbb{R})$ is the (symmetric, positive definite) {\bf covariance matrix}.

\begin{center}
\begin{figure}
\includegraphics[scale=0.15]{images/multinormal_density.png} 
\includegraphics[scale=0.15]{images/no_multinormal_density.png} 
\caption{Left: Multivariate Normal Distribution, Right: Non-Multivariate Normal Distribution}
\end{figure}
\end{center}
\end{frame}



\begin{frame}{References}{Slides and notebook available at juanitorduz.github.io}
\bibliographystyle{alpha}
\bibliography{references} 
\end{frame}

\end{document}





























