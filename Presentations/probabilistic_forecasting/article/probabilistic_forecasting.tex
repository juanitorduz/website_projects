\documentclass[11pt]{amsart}
\usepackage[all]{xy}
\usepackage[dvips]{graphicx}
\usepackage{amsfonts}
\usepackage{amssymb,latexsym,amsmath}
\usepackage{amsthm}
\usepackage{color}
\usepackage{empheq}
\usepackage{float}
\usepackage{hyperref}
\usepackage{listings}
\usepackage{mathrsfs}
\usepackage{slashed}
\usepackage{tikz}
\usepackage{svg}

\definecolor{dkgreen}{rgb}{0,0.6,0}
\definecolor{gray}{rgb}{0.5,0.5,0.5}
\definecolor{mauve}{rgb}{0.58,0,0.82}

\theoremstyle{definition}
\newtheorem{remark}{Remark}

\lstset{frame=tb,
    language=python,
    aboveskip=10pt,
    belowskip=5pt,
    showstringspaces=false,
    columns=flexible,
    basicstyle={\small\ttfamily},
    numbers=left,
    numberstyle=\tiny\color{gray},
    keywordstyle=\color{blue},
    commentstyle=\color{dkgreen},
    stringstyle=\color{mauve},
    breaklines=true,
    breakatwhitespace=true,
    tabsize=3,
    captionpos=b,
    belowcaptionskip=10pt
}

\textwidth = 420pt
\oddsidemargin = 18pt
\evensidemargin = 18pt

\begin{document}
\title{Probabilistic Time Series Forecasting with NumPyro: A Practical Guide}
\author{Juan Camilo Orduz}
\email{juanitorduz@gmail.com}
\urladdr{\href{https://juanitorduz.github.io/}{https://juanitorduz.github.io/}}
\address{Berlin, Germany}
\date{\today}

\begin{abstract}
    This article presents a comprehensive guide to probabilistic time series forecasting using NumPyro, a lightweight
    probabilistic programming library built on JAX. We demonstrate a progression from simple exponential smoothing models
    to sophisticated hybrid deep state-space models, emphasizing practical implementation patterns and scalability
    considerations. Through concrete examples, we illustrate key concepts including the \texttt{scan} function for
    recursive relationships, hierarchical model structures for sharing information across time series, custom likelihoods
    for censored and intermittent demand data, and model calibration techniques using additional likelihood terms. Two
    showcase applications---availability-constrained TSB models and Hilbert Space Gaussian Process calibration---highlight
    the flexibility of probabilistic programming for encoding domain knowledge directly into forecasting models.
    The integration with the broader JAX ecosystem, including Optax for optimization and Flax for neural network
    components, enables scalable inference via stochastic variational inference (SVI) on thousands of time series.
\end{abstract}

\maketitle

\tableofcontents

\addtocontents{toc}{\protect\setcounter{tocdepth}{2}}

% =============================================================================
% SECTION 1: INTRODUCTION
% =============================================================================
\section{Introduction}

Forecasting is a critical component of business planning across industries---from retail inventory management
to financial market analysis, from energy demand prediction to marketing budget allocation. Traditionally,
businesses have relied on point forecasts that provide a single estimate of future values. While these
approaches can work reasonably well under stable conditions, they often fail to capture the inherent
uncertainty in real-world systems and can lead to suboptimal decisions when confronted with volatile
or complex environments.

Consider a retailer deciding how much inventory to stock. A point forecast of 100 units tells them
nothing about whether true demand might be 80 or 120. If they stock exactly 100 units and demand
turns out to be 120, they lose 20 sales. If demand is only 80, they're stuck with 20 units of excess
inventory. The \emph{cost} of these errors is often asymmetric---lost sales may damage customer
relationships permanently, while excess inventory merely ties up capital. Without understanding the
full distribution of possible outcomes, the retailer cannot make an informed decision that accounts
for these asymmetric risks.

\subsection*{The Paradigm Shift: From Point Estimates to Distributions}

Probabilistic forecasting addresses these limitations by generating complete probability distributions
over possible future outcomes rather than single-point predictions. This paradigm shift provides
decision-makers with a more comprehensive view of potential scenarios, enabling robust planning
that accounts for risk and uncertainty.

But probabilistic forecasting offers more than just uncertainty quantification. As we will demonstrate
throughout this article, probabilistic programming frameworks enable us to directly incorporate
business constraints, domain knowledge, and even causal relationships into our models. When sales
data is censored by stockouts, we can model the censoring mechanism explicitly. When we have
experimental results from lift tests, we can inject that knowledge as calibration constraints.
When different product categories share common seasonality patterns, we can let them borrow
statistical strength from each other through hierarchical structures.

This flexibility is what sets probabilistic forecasting apart from traditional methods. Classical
approaches treat the model as a black box that maps inputs to outputs. Probabilistic programming
inverts this relationship: we specify \emph{how we believe the data was generated}, and the
inference machinery works backward to learn the parameters consistent with our observations.
This generative perspective makes it natural to encode domain expertise directly into the model
structure.

\subsection*{NumPyro and the JAX Ecosystem}

NumPyro \cite{phan2019composable} is a lightweight probabilistic programming library that provides
a NumPy backend for Pyro \cite{bingham2019pyro}. It relies on JAX \cite{jax2018github} for automatic
differentiation and JIT compilation to GPU/CPU, enabling efficient inference on large-scale problems.
This combination of expressiveness and performance makes NumPyro particularly well-suited for time
series forecasting applications where we may need to fit models to thousands of related time series.

The integration with the broader JAX ecosystem---including Optax for gradient-based optimization
and Flax for neural network components---enables sophisticated hybrid architectures that combine
the interpretability of state-space models with the flexibility of deep learning. Modern inference
algorithms like Stochastic Variational Inference (SVI) allow scaling to datasets that would be
computationally infeasible with traditional Markov Chain Monte Carlo methods.

\subsection*{Article Overview}

This article presents a progression from foundational concepts to advanced techniques, with emphasis
on the \emph{motivation} behind each approach and the business problems they solve. We begin with
the \texttt{scan} function, the computational building block that enables efficient implementation
of recursive time series models. We then develop exponential smoothing and ARIMA models, demonstrating
the basic patterns for state-space modeling in NumPyro.

The middle sections address specialized challenges that arise in real-world forecasting: hierarchical
models for sharing information across related time series, intermittent demand models for products
with sporadic sales, and censored demand models for situations where stockouts mask true demand.
Two showcase sections demonstrate the unique power of probabilistic programming: an availability-constrained
TSB model that encodes business logic directly into the transition function, and a model calibration
technique that injects domain expertise through additional likelihood terms.

We conclude with stochastic variational inference for scalable training and hybrid deep state-space
models that integrate neural network components for learning complex patterns across many time series.

% =============================================================================
% SECTION 2: LITERATURE REVIEW
% =============================================================================
\section{Literature Review and Related Work}

Probabilistic forecasting has a rich history spanning classical statistical methods and modern deep learning approaches.
This section surveys key developments and positions the NumPyro-based approach within this landscape.

\subsection{Classical Statistical Methods}

\textbf{State space models and ETS}: Hyndman et al. \cite{hyndman2002state, hyndman2008forecasting} reformulated
exponential smoothing as innovations state space models, enabling likelihood estimation and automatic model selection.
The ETS (Error-Trend-Seasonality) taxonomy provides a systematic framework for exponential smoothing variants.

\textbf{ARIMA family}: The Box-Jenkins methodology established autoregressive integrated moving average models as
a cornerstone of time series analysis. Extensions include seasonal ARIMA (SARIMA) and vector autoregressive (VAR)
models for multivariate forecasting.

\textbf{Structural time series}: Harvey \cite{harvey1989forecasting} and Durbin \& Koopman \cite{durbin2012time}
developed comprehensive frameworks for state space modeling with Kalman filtering, providing the theoretical
foundation for many modern approaches.

\subsection{Deep Learning Approaches}

\textbf{DeepAR} \cite{salinas2020deepar}: A seminal work in probabilistic deep learning for forecasting, DeepAR
trains a global autoregressive RNN model across many related time series. The key innovation is learning to share
patterns across series while producing probabilistic forecasts through parametric output distributions.

\textbf{Temporal Fusion Transformers} \cite{lim2021temporal}: Attention-based architectures for multi-horizon
forecasting with interpretable attention weights and handling of static and dynamic covariates.

\textbf{N-BEATS and N-HiTS} \cite{oreshkin2020nbeats}: Pure deep learning approaches without time series-specific
components, using basis expansion for interpretability.

\subsection{Probabilistic Programming Frameworks}

\textbf{Pyro Forecasting Module}: Built on Pyro/PyTorch, this module provides hierarchical models with global-local
structure and SVI for scalable inference. The Dynamic Linear Model tutorial \cite{pyro_dlm_tutorial} demonstrates
powerful calibration techniques that we extend in this work.

\textbf{Prophet} \cite{taylor2018forecasting}: Facebook's decomposable model combining trend, seasonality, and
holiday effects, designed for business forecasting at scale with Stan backend for uncertainty quantification.

\textbf{GluonTS} \cite{alexandrov2020gluonts}: A unified Python interface for probabilistic time series modeling,
implementing DeepAR, Transformers, and many other methods.

\subsection{Recent Advances}

\textbf{LGT/SGT Models} \cite{smyl2024local}: Bayesian exponential smoothing with flexible trend components
between linear and exponential, using Student-t errors for robustness.

\textbf{ADAM} \cite{svetunkov2023adam}: The Augmented Dynamic Adaptive Model provides a comprehensive state
space framework with extensive treatment of intermittent demand and censored observations.

\subsection{Gap Addressed by This Work}

Existing tools often lack easy customization of model components, direct integration of domain knowledge,
and transparent model specification. NumPyro addresses these gaps by providing:

\begin{itemize}
    \item Composable model building with \texttt{scan} for time series
    \item Custom likelihoods (censored, zero-inflated, availability-constrained)
    \item JAX ecosystem integration for GPU acceleration
    \item SVI for scalable inference on thousands of series
    \item Neural network integration via Flax NNX
\end{itemize}

% =============================================================================
% SECTION 3: THE SCAN FUNCTION
% =============================================================================
\section{Technical Foundation: The Scan Function}

Time series models are fundamentally about \emph{sequential dependence}---each observation depends on what
came before it. The current sales level depends on yesterday's level. Today's forecast error depends on
yesterday's forecast error. The seasonal pattern at time $t$ depends on the seasonal pattern from one
period ago. This recursive structure, where the present depends on the past, is what distinguishes time
series from cross-sectional data.

In most programming languages, we would express this dependence with a simple \texttt{for} loop. But
\texttt{for} loops are problematic in the context of automatic differentiation and GPU computation.
They execute sequentially in Python, preventing parallelization, and frameworks like JAX cannot
efficiently compute gradients through them. The \texttt{scan} function solves this problem by
expressing sequential computation in a form that JAX can compile and differentiate efficiently.

The \texttt{scan} function takes a transition function that describes how to move from one time step
to the next, an initial state, and a sequence of inputs. It returns the final state and all intermediate
outputs---exactly what we need for time series forecasting where we want to track latent states
(level, trend, seasonality) and generate predictions at each step.

\begin{lstlisting}[caption={Pure Python implementation of the scan function}]
def scan(f, init, xs):
    """Pure Python implementation of scan.

    Parameters
    ----------
    f : callable
        A function to be scanned: (carry, x) -> (carry, y)
    init : any
        Initial loop carry value
    xs : array
        Values over which to scan along leading axis
    """
    carry = init
    ys = []
    for x in xs:
        carry, y = f(carry, x)
        ys.append(y)
    return carry, np.stack(ys)
\end{lstlisting}

The transition function \texttt{f} takes the current carry state and input, returning an updated carry and
output. This pattern naturally maps to state-space model formulations where the carry represents latent states
(level, trend, seasonality) and the output represents predictions.

\begin{remark}
    In JAX, \texttt{jax.lax.scan} compiles the loop efficiently, avoiding Python overhead and enabling
    automatic differentiation through the entire sequence. NumPyro's \texttt{contrib.control\_flow.scan}
    extends this with proper handling of probabilistic primitives.
\end{remark}

% =============================================================================
% SECTION 4: EXPONENTIAL SMOOTHING
% =============================================================================
\section{Exponential Smoothing Models}

Consider a retailer whose sales patterns shift gradually over time. Last month's average sales provide
useful information about this month's expected level, but not perfect information---consumer preferences
evolve, competitors enter the market, and economic conditions change. The retailer faces a fundamental
tradeoff: rely too heavily on recent observations and the forecast becomes noisy, overreacting to
random fluctuations; rely too heavily on distant history and the forecast becomes sluggish, slow to
adapt to genuine changes in the underlying pattern.

Exponential smoothing elegantly resolves this tradeoff through a single parameter $\alpha$ that controls
the balance between \emph{memory} and \emph{responsiveness}. When $\alpha$ is close to 1, the model
places most weight on the most recent observation, adapting quickly to changes but potentially
overreacting to noise. When $\alpha$ is close to 0, the model averages over a long history, providing
stability but potentially missing genuine shifts in the pattern.

\subsection{The ETS Framework}

Modern exponential smoothing is unified under the ETS (Error, Trend, Seasonality) framework, popularized
by Hyndman et al. \cite{hyndman2008forecasting}. This framework classifies models based on how they
handle three components:
\begin{itemize}
    \item \textbf{Error}: How the observation noise enters the model (Additive or Multiplicative).
    \item \textbf{Trend}: The long-term progression of the series (None, Additive, or Additive Damped).
    \item \textbf{Seasonality}: Recurring patterns over fixed periods (None, Additive, or Multiplicative).
\end{itemize}

By viewing exponential smoothing as an \emph{innovations state space model}, we move beyond simple
recursive formulas to a probabilistic generative model. This perspective is what allows us to
implement these methods in NumPyro, utilizing likelihood-based estimation and obtaining full
predictive distributions.

\subsection{Simple Exponential Smoothing}

The level equations for simple exponential smoothing capture this intuition mathematically:
\begin{align*}
    \hat{y}_{t+h|t} & = l_t                                 \\
    l_t             & = \alpha y_t + (1 - \alpha) l_{t-1}
\end{align*}

where $y_t$ is the observed value, $l_t$ is the level, and $\alpha \in (0, 1)$ is the smoothing parameter.

\begin{figure}[H]
    \centering
    \includegraphics[width=\textwidth]{images/exponential_smoothing_numpyro_16_1.png}
    \caption{Example time series data for exponential smoothing. The data exhibits a clear trend component
        that simple exponential smoothing must capture through the level dynamics.}
    \label{fig:exp_smooth_data}
\end{figure}

The NumPyro implementation uses the \texttt{scan} pattern:

\begin{lstlisting}[caption={Transition function for simple exponential smoothing}]
def transition_fn(carry, t):
    previous_level = carry

    level = jnp.where(
        t < t_max,
        level_smoothing * y[t] + (1 - level_smoothing) * previous_level,
        previous_level,  # Forecast: no update
    )

    mu = previous_level
    pred = numpyro.sample("pred", dist.Normal(loc=mu, scale=noise))

    return level, pred
\end{lstlisting}

The full model wraps this transition function with appropriate priors. We place a $\text{Beta}(1, 1)$
prior on the smoothing parameter $\alpha$, allowing the data to determine how quickly the model
should adapt to new observations. The initial level receives a weakly informative normal prior,
and the observation noise receives a half-normal prior ensuring positivity. NumPyro's \texttt{scan}
function then iterates the transition function through time, with \texttt{handlers.condition}
linking the model's predictions to the observed data.

\begin{figure}[H]
    \centering
    \includegraphics[width=\textwidth]{images/exponential_smoothing_numpyro_29_1.png}
    \caption{Posterior distributions of model parameters. The smoothing parameter $\alpha$ concentrates
        around 0.8, indicating relatively fast adaptation to recent observations.}
    \label{fig:exp_smooth_params}
\end{figure}

\begin{figure}[H]
    \centering
    \includegraphics[width=\textwidth]{images/exponential_smoothing_numpyro_36_1.png}
    \caption{In-sample fit and forecast with 94\% highest density intervals. The model captures the
        underlying trend while appropriately quantifying forecast uncertainty.}
    \label{fig:exp_smooth_forecast}
\end{figure}

\subsection{Extensions: Trend, Seasonality, and Damping}

The exponential smoothing framework extends naturally to include trend and seasonal components. The
Holt-Winters model with damped trend combines:

\begin{align*}
    \hat{y}_{t+h|t} & = l_t + (\phi + \phi^2 + \cdots + \phi^h) b_t + s_{t+h-m(k+1)} \\
    l_t             & = \alpha (y_t - s_{t-m}) + (1 - \alpha)(l_{t-1} + \phi b_{t-1}) \\
    b_t             & = \beta (l_t - l_{t-1}) + (1 - \beta) \phi b_{t-1}              \\
    s_t             & = \gamma (y_t - l_{t-1} - \phi b_{t-1}) + (1 - \gamma) s_{t-m}
\end{align*}

where $b_t$ is the trend, $s_t$ is the seasonal component, $\phi$ is the damping factor, and $m$ is the
seasonal period.

\begin{figure}[H]
    \centering
    \includegraphics[width=\textwidth]{images/exponential_smoothing_numpyro_78_1.png}
    \caption{Damped trend seasonal model forecast. The model captures both the trend decay and seasonal
        patterns, with uncertainty bands widening appropriately into the forecast horizon.}
    \label{fig:exp_smooth_damped}
\end{figure}

% =============================================================================
% SECTION 5: ARIMA MODELS
% =============================================================================
\section{ARIMA Models}

While exponential smoothing models the \emph{level} of a series directly, ARIMA models take a different
perspective: they model the \emph{autocorrelation structure} of the data. The key insight is that
many time series exhibit predictable patterns in how current values relate to past values and past
forecast errors.

\subsection{Stationarity, Differencing, and Model Specification}

A fundamental concept in ARIMA modeling is \emph{stationarity}. A stationary time series is one whose
statistical properties---mean, variance, and autocorrelation---are constant over time. Most real-world
time series, however, exhibit trends or seasonality and are therefore non-stationary.

The ``I'' in ARIMA stands for \emph{Integrated}, referring to the process of \emph{differencing}
the data to achieve stationarity. If a series has a trend, we take the first difference:
$y'_t = y_t - y_{t-1}$. If it has a seasonal pattern, we might take a seasonal difference.

An ARIMA model is typically specified by three parameters $(p, d, q)$:
\begin{itemize}
    \item $p$: The order of the \textbf{autoregressive} (AR) part. It tells us how many lagged
          observations to include as predictors.
    \item $d$: The degree of \textbf{differencing} required to make the series stationary.
    \item $q$: The order of the \textbf{moving average} (MA) part. It tells us how many lagged
          forecast errors to include as predictors.
\end{itemize}

In practice, we use tools like the Autocorrelation Function (ACF) and Partial Autocorrelation
Function (PACF) to identify the appropriate values for $p$ and $q$. In the NumPyro framework,
these models follow the same \texttt{scan} pattern as exponential smoothing, demonstrating the
flexibility of this computational approach.

\subsection{ARMA(1,1) Model}

The ARMA(1,1) model combines one autoregressive and one moving average term:
\begin{equation*}
    y_t = \mu + \phi y_{t-1} + \theta \varepsilon_{t-1} + \varepsilon_t
\end{equation*}

\begin{lstlisting}[caption={Transition function for ARMA(1,1) model}]
def transition_fn(carry, t):
    y_prev, error_prev = carry
    ar_part = phi * y_prev
    ma_part = theta * error_prev
    pred = mu + ar_part + ma_part
    error = y[t] - pred
    return (y[t], error), error
\end{lstlisting}

\begin{figure}[H]
    \centering
    \includegraphics[width=0.9\textwidth]{images/arma_numpyro_40_1.png}
    \caption{ARMA(1,1) model fit showing the in-sample predictions and residuals. The model effectively
        captures the short-term autocorrelation structure in the data.}
    \label{fig:arma_fit}
\end{figure}

\subsection{VAR Models and Impulse Response Functions}

Vector autoregressive (VAR) models extend ARMA to multivariate settings, enabling analysis of dynamic
relationships between multiple time series. A key application is computing impulse response functions
(IRFs) that trace the effect of a shock to one variable through the system.

\begin{figure}[H]
    \centering
    \includegraphics[width=0.8\textwidth]{images/var_irf.png}
    \caption{Impulse response functions from a Bayesian VAR model. The posterior distributions over
        IRFs quantify uncertainty in the dynamic relationships between variables.}
    \label{fig:var_irf}
\end{figure}

While the models discussed so far are powerful, fitting them to large datasets with many related time
series poses significant computational challenges. In the next section, we introduce the inference
machinery that allows us to scale these probabilistic models to thousands of series.

% =============================================================================
% SECTION 6: STOCHASTIC VARIATIONAL INFERENCE
% =============================================================================
\section{Stochastic Variational Inference}

Markov Chain Monte Carlo (MCMC) can be computationally expensive for large-scale forecasting problems
involving thousands of time series. Stochastic Variational Inference (SVI) \cite{blei2017variational}
transforms posterior inference into an optimization problem, enabling scalable inference. Establishing
this foundation early is critical for understanding how we scale the more complex models presented
in the subsequent sections.

\subsection{Core Concepts}

\textbf{Variational family}: Approximate the complex posterior $p(\theta | y)$ with a simpler
distribution $q_\phi(\theta)$ parameterized by $\phi$.

\textbf{ELBO}: The Evidence Lower Bound serves as the optimization target:
\begin{equation*}
    \text{ELBO}(\phi) = \mathbb{E}_{q_\phi}[\log p(y | \theta)] - \text{KL}[q_\phi(\theta) \| p(\theta)]
\end{equation*}

The first term encourages the approximation to explain the data well; the second term regularizes
toward the prior.

\subsection{NumPyro SVI Workflow}

\begin{lstlisting}[caption={SVI setup with Optax optimizer}]
from numpyro.infer import SVI, Trace_ELBO
from numpyro.infer.autoguide import AutoNormal
import optax

# Define guide (variational family)
guide = AutoNormal(model)

# Define optimizer with Optax
scheduler = optax.linear_onecycle_schedule(
    transition_steps=num_steps,
    peak_value=0.01,
    pct_start=0.1
)
optimizer = optax.chain(
    optax.clip_by_global_norm(1.0),
    optax.adam(learning_rate=scheduler)
)

# Initialize SVI
svi = SVI(model=model, guide=guide, optim=optimizer, loss=Trace_ELBO())

# Run optimization
svi_result = svi.run(rng_key, num_steps, y=data)
\end{lstlisting}

\subsection{JAX Ecosystem Integration}

NumPyro integrates seamlessly with the broader JAX ecosystem:

\begin{itemize}
    \item \textbf{Optax}: Composable gradient transformations including Adam, learning rate schedulers,
          and gradient clipping.
    \item \textbf{Flax NNX}: Neural network modules that can be integrated into probabilistic models
          (see Section~\ref{sec:hybrid}).
    \item \textbf{JAX JIT}: Automatic compilation for efficient execution on GPU/TPU.
\end{itemize}

\subsection{AutoGuides}

NumPyro provides automatic guide construction:

\begin{itemize}
    \item \texttt{AutoNormal}: Mean-field approximation with independent Normal for each parameter
    \item \texttt{AutoMultivariateNormal}: Full covariance approximation
    \item \texttt{AutoLowRankMultivariateNormal}: Low-rank approximation for high-dimensional problems
\end{itemize}

The automatic handling of constraints and transformations makes SVI accessible without deep
expertise in variational inference. With this scalable inference toolset in place, we now turn
to models that exploit relationships across multiple time series, beginning with the foundational
concept of hierarchical modeling.

% =============================================================================
% SECTION 7: HIERARCHICAL MODELS
% =============================================================================
\section{Hierarchical Models}

Imagine you are forecasting tourism volumes for 308 combinations of Australian states, regions, and
travel purposes. Some combinations---say, business travel to Sydney---have abundant historical data.
Others---perhaps holiday travel to a small regional town---have only a handful of observations.
How should you approach this problem?

One option is \emph{complete pooling}: fit a single model to all data, assuming all regions behave
identically. This approach gains statistical power but ignores meaningful regional differences.
Another option is \emph{no pooling}: fit a separate model to each region independently. This
respects heterogeneity but suffers from data sparsity---how can you reliably estimate seasonality
for a region with only 8 quarterly observations?

Hierarchical models offer a third way: \emph{partial pooling}. The key insight is that related
time series often share common structure while exhibiting individual variation. All Australian
regions might share a similar seasonal pattern (more tourism in summer), but the amplitude and
timing may differ. By modeling these shared patterns explicitly, hierarchical models allow
data-rich series to ``lend strength'' to data-poor series, improving estimates across the board.

This is where Bayesian methods truly shine. The hierarchical structure is naturally expressed
through the prior distribution: individual parameters are drawn from a population distribution
whose parameters are themselves unknown and learned from data. Classical frequentist methods
have no natural analog to this information-sharing mechanism.

\subsection{Hierarchical Exponential Smoothing}

We demonstrate this concept using the quarterly Australian tourism dataset, where visitor counts
are organized by state, region, and purpose of travel.

\begin{figure}[H]
    \centering
    \includegraphics[width=\textwidth]{images/hierarchical_exponential_smoothing_45_0.png}
    \caption{Australian tourism data showing multiple time series organized hierarchically by region.
        The data exhibits both shared seasonal patterns and region-specific characteristics.}
    \label{fig:hier_data}
\end{figure}

The hierarchical structure places priors on the smoothing parameters that allow information sharing:

\begin{align*}
    \alpha_i                       & \sim \text{Beta}(\mu_\alpha \kappa, (1-\mu_\alpha) \kappa) \\
    \mu_\alpha                     & \sim \text{Beta}(2, 2)                                     \\
    \kappa                         & \sim \text{HalfNormal}(10)
\end{align*}

where $\mu_\alpha$ is the population mean smoothing parameter and $\kappa$ controls the concentration
around this mean.

\begin{figure}[H]
    \centering
    \includegraphics[width=\textwidth]{images/hierarchical_exponential_smoothing_47_0.png}
    \caption{Posterior distributions of smoothing parameters across regions. The hierarchical structure
        shrinks extreme estimates toward the population mean while preserving meaningful heterogeneity.}
    \label{fig:hier_params}
\end{figure}

\begin{figure}[H]
    \centering
    \includegraphics[width=0.85\textwidth]{images/hierarchical_exponential_smoothing_69_0.png}
    \caption{Hierarchical forecasts for selected regions with 94\% HDIs. The model appropriately
        quantifies uncertainty, with wider intervals for regions with higher volatility.}
    \label{fig:hier_forecast}
\end{figure}

\subsection{Baseline Production Model}

For large-scale applications, we developed a baseline model that serves as a robust production-ready
solution. This model combines local level dynamics with seasonal features and external covariates,
structured as a hierarchical state-space model.

\subsubsection{Mathematical Structure}

The model for each time series $i$ at time $t$ is formulated as:
\begin{align*}
    y_{i,t} & \sim \text{Normal}(\mu_{i,t}, \sigma) \\
    \mu_{i,t} & = l_{i,t} + s_{i,t} + \mathbf{x}_{i,t}^\top \boldsymbol{\beta}_i \\
    l_{i,t} & = l_{i,t-1} + \delta_{i,t} \\
    \delta_{i,t} & \sim \text{Normal}(0, \sigma_\delta)
\end{align*}
where $l_{i,t}$ is the local level (a random walk), $s_{i,t}$ is the seasonal component, and
$\mathbf{x}_{i,t}^\top \boldsymbol{\beta}_i$ represents the effect of external covariates
(e.g., promotions, holidays).

\subsubsection{Implementation Details in NumPyro}

A key challenge in fitting hierarchical models like this is the geometry of the posterior distribution,
often referred to as ``Neal's funnel.'' When the group-level variance $\sigma_\delta$ is small, the
drift parameters $\delta_{i,t}$ become highly constrained, making it difficult for inference
algorithms to explore the parameter space effectively.

To address this, we use \emph{non-centered parameterization}. Instead of sampling $\delta_{i,t}$
directly from $\text{Normal}(0, \sigma_\delta)$, we sample a standardized variable $\tilde{\delta}_{i,t} \sim \text{Normal}(0, 1)$
and then set $\delta_{i,t} = \tilde{\delta}_{i,t} \cdot \sigma_\delta$. NumPyro provides the
\texttt{LocScaleReparam} handler \cite{gorinova2019automatic} to automate this transformation:

\begin{lstlisting}[caption={Hierarchical drift with LocScaleReparam}]
with numpyro.handlers.reparam(config={"drift": LocScaleReparam(centered=0)}):
    drift = numpyro.sample("drift", dist.Normal(0, drift_scale))
\end{lstlisting}

This reparameterization is critical for both MCMC stability and SVI convergence. The model
utilizes global-local structure where scales like $\sigma_\delta$ and $\sigma$ are shared across
all series, while the drift and seasonal components are local. This allows the model to
``borrow strength'' across the entire dataset while remaining flexible enough to capture
individual series behavior \cite{orduz_hierarchical_1}.

\begin{figure}[H]
    \centering
    \includegraphics[width=0.8\textwidth]{images/baseline_covariates.png}
    \caption{Production baseline model architecture. The local level captures trend shifts via a random
        walk, while Fourier modes or repeated features handle seasonality. External covariates provide
        the model with information about planned interventions like promotions.}
    \label{fig:baseline}
\end{figure}

The hierarchical models discussed so far assume a continuous demand stream. However, in
many retail environments, we encounter products with sporadic sales, requiring a shift in how we
model the demand generation process.

% =============================================================================
% SECTION 8: INTERMITTENT DEMAND
% =============================================================================
\section{Intermittent Demand Forecasting}

In retail, intermittent time series are the norm rather than the exception. While top-selling products
might show regular daily sales patterns, the vast majority of items in a typical retail catalog exhibit
sporadic demand---many days showing zero sales followed by occasional purchases. This pattern is
particularly common for niche products, seasonal items, spare parts, or products with long replacement
cycles. A hardware store might sell a specialized plumbing fitting once every two weeks; an auto parts
retailer might see demand for a particular gasket only a few times per month.

Standard forecasting methods, designed for continuous demand streams, struggle with these patterns.
Exponential smoothing applied directly to intermittent data will produce forecasts that are pulled
toward zero by the many zero observations, systematically underestimating demand when it does occur.
The fundamental problem is that these methods conflate two distinct phenomena: \emph{whether} demand
will occur and \emph{how much} demand will occur given that it happens.

The insight behind specialized intermittent demand methods is to separate these two questions.
Instead of modeling the raw demand series directly, we decompose it into a \emph{demand size}
component (how much is demanded when demand occurs) and a \emph{demand occurrence} component
(how often demand occurs). Each component can then be forecast using standard techniques, and
the final forecast is their product.

\subsection{Croston's Method}

Croston's method \cite{croston1972forecasting} pioneered this decomposition approach. The idea is
simple: extract the non-zero demand values into a ``demand size'' series $z_t$, and extract the
intervals between demand occurrences into an ``inter-arrival period'' series $p_t$. Apply exponential
smoothing to each series separately, then combine the forecasts as $\hat{y}_{t+h} = \hat{z}_{t+h} / \hat{p}_{t+h}$.

\begin{figure}[H]
    \centering
    \includegraphics[width=\textwidth]{images/croston_numpyro_6_1.png}
    \caption{Example intermittent demand series showing sporadic non-zero values. The long stretches
        of zeros interspersed with variable demand sizes motivate the separation approach.}
    \label{fig:croston_data}
\end{figure}

\begin{figure}[H]
    \centering
    \includegraphics[width=\textwidth]{images/croston_numpyro_10_1.png}
    \caption{Decomposition into demand size (top) and inter-arrival time (bottom) series.
        Each component can now be modeled with standard exponential smoothing.}
    \label{fig:croston_decomp}
\end{figure}

The NumPyro implementation uses \texttt{scope} to combine two exponential smoothing models:

\begin{lstlisting}[caption={Croston's method using scoped models}]
def croston_model(z: ArrayLike, p_inv: ArrayLike, future: int = 0) -> None:
    z_forecast = scope(level_model, "demand")(z, future)
    p_inv_forecast = scope(level_model, "period_inv")(p_inv, future)

    if future > 0:
        numpyro.deterministic("z_forecast", z_forecast)
        numpyro.deterministic("p_inv_forecast", p_inv_forecast)
        numpyro.deterministic("forecast", z_forecast * p_inv_forecast)
\end{lstlisting}

\begin{figure}[H]
    \centering
    \includegraphics[width=0.9\textwidth]{images/croston_numpyro_39_1.png}
    \caption{Croston's method forecast with uncertainty quantification. The probabilistic
        formulation provides credible intervals that appropriately reflect the high uncertainty
        inherent in intermittent demand forecasting.}
    \label{fig:croston_forecast}
\end{figure}

\subsection{TSB Method}

The Teunter-Syntetos-Babai (TSB) method \cite{teunter2011intermittent} takes a slightly different
approach: instead of modeling inter-arrival periods, it directly models the \emph{probability} that
demand occurs in each period. This is particularly advantageous because it updates the demand
probability at \emph{every} time step, not just when demand occurs.

The TSB model maintains two latent states: the demand size $z_t$ and the demand probability $p_t$.
The update equations are:
\begin{align*}
    z_t & = \begin{cases} 
                \alpha y_t + (1 - \alpha) z_{t-1} & \text{if } y_t > 0 \\
                z_{t-1} & \text{if } y_t = 0
            \end{cases} \\
    p_t & = \beta I(y_t > 0) + (1 - \beta) p_{t-1}
\end{align*}
where $I(\cdot)$ is the indicator function. Notice that when $y_t = 0$, the probability $p_t$
decays by a factor of $(1 - \beta)$. This allows the model to quickly adapt to obsolescence
or long periods of zero demand. The forecast is then the product of the two components:
$\hat{y}_{t+h} = \hat{z}_{t+h} \cdot \hat{p}_{t+h}$.

This formulation often provides more stable forecasts and forms the foundation for the
availability-constrained model we develop in the next section.

\begin{figure}[H]
    \centering
    \includegraphics[width=0.8\textwidth]{images/tsb_numpyro_36_1.png}
    \caption{TSB method cross-validation results showing one-step-ahead forecasts. The probability
        formulation often provides more stable forecasts than the period-based Croston approach.}
    \label{fig:tsb_cv}
\end{figure}

\subsection{Zero-Inflated TSB Model}

For count data with excess zeros, we can modify the TSB model to use a zero-inflated negative
binomial distribution:

\begin{lstlisting}[caption={Zero-inflated TSB transition function}]
def transition_fn(carry, t):
    z_prev, p_prev = carry
    z_next = ...  # Demand size update
    p_next = ...  # Probability update

    mu = z_next
    gate = 1 - p_next
    pred = numpyro.sample(
        "pred",
        dist.ZeroInflatedNegativeBinomial2(
            mean=mu, concentration=concentration, gate=gate
        ),
    )
    return (z_next, p_next), pred
\end{lstlisting}

\begin{figure}[H]
    \centering
    \includegraphics[width=\textwidth]{images/zi_tsb_numpyro_35_1.png}
    \caption{Zero-inflated TSB model cross-validation. The explicit zero-inflation component
        better captures the two-stage nature of intermittent demand: first whether demand occurs,
        then the demand size conditional on occurrence.}
    \label{fig:zi_tsb}
\end{figure}

Building on the TSB framework, we can now address one of the most common challenges in
retail forecasting: accounting for the impact of product availability on observed demand patterns.

% =============================================================================
% SECTION 9: AVAILABILITY-CONSTRAINED TSB (SHOWCASE)
% =============================================================================
\section{Availability-Constrained TSB: A Custom Probabilistic Model}

This section showcases one of the most powerful capabilities of probabilistic programming: the
ability to encode domain knowledge directly into the model structure through a simple modification
to the transition function. The solution we present here would be difficult or impossible to
implement in a traditional forecasting package, yet in NumPyro it requires changing just a few
lines of code.

\subsection{The Problem: Why Zeros Happen}

As discussed by Svetunkov \cite{svetunkov_zeros}, zeros in intermittent time series can arise
from two fundamentally different causes. A zero might represent \emph{true zero demand}---no
customer wanted the product that day. Or it might represent an \emph{availability constraint}---the
product was out of stock, so no sales were recorded even though customers may have wanted to buy.

This distinction has profound implications for forecasting. If a product shows three consecutive
zeros because nobody wanted it, the model should reasonably conclude that demand probability has
declined---future forecasts should be lower. But if those three zeros occurred because the product
was out of stock, the model should \emph{not} update its demand probability at all. The zeros
contain no information about underlying demand; they merely reflect a supply constraint.

Standard TSB models cannot make this distinction. They treat all zeros equally, causing the estimated
demand probability to decay regardless of the cause. The business consequence is severe: after a
stockout, the model systematically underforecasts future demand. This creates a vicious cycle where
low forecasts lead to low inventory orders, which lead to more stockouts, which further suppress
forecasts. The retailer may conclude that a product is dying when in fact it remains popular---they
just keep running out of it.

\begin{figure}[H]
    \centering
    \includegraphics[width=\textwidth]{images/availability_tsb_12_0.png}
    \caption{Simulated data with availability mask. The red regions indicate periods of
        unavailability where zeros are not informative about true demand.}
    \label{fig:avail_data}
\end{figure}

\subsection{The Elegant Mathematical Modification}

The key insight is to modify the probability update equation to account for availability:

\textbf{Standard TSB} (when $y_t = 0$):
\begin{align*}
    z_{t+1} & = z_t                \\
    p_{t+1} & = (1 - \beta) \cdot p_t
\end{align*}

\textbf{Availability-Constrained TSB} (when $y_t = 0$):
\begin{align*}
    z_{t+1} & = z_t                         \\
    p_{t+1} & = (1 - a_t \cdot \beta) \cdot p_t
\end{align*}

where $a_t \in \{0, 1\}$ indicates availability. When $a_t = 0$ (unavailable), the probability
does not decay. Additionally, the forecast incorporates availability:
\begin{equation*}
    \hat{y}_{t+1} = a_t \cdot z_t \cdot p_t
\end{equation*}

\begin{lstlisting}[caption={Availability-constrained TSB transition function}]
def transition_fn(carry, t):
    z_prev, p_prev = carry

    z_next = jnp.where(
        counts[t] > 0,
        z_smoothing * counts[t] + (1 - z_smoothing) * z_prev,
        z_prev,
    )

    p_next = jnp.where(
        counts[t] > 0,
        p_smoothing + (1 - p_smoothing) * p_prev,
        (1 - available[t] * p_smoothing) * p_prev,  # Key modification
    )

    mu = z_next * p_next
    pred = numpyro.sample("pred", dist.Normal(loc=mu, scale=noise))
    pred = numpyro.deterministic("pred_obs", available[t] * pred)

    return (z_next, p_next), pred
\end{lstlisting}

\begin{figure}[H]
    \centering
    \includegraphics[width=\textwidth]{images/availability_tsb_30_0.png}
    \caption{Key result: forecast comparison after zeros with availability=0. The standard TSB
        (left) incorrectly decays the demand probability, while the availability-constrained
        model (right) maintains the probability, correctly anticipating demand recovery.}
    \label{fig:avail_result}
\end{figure}

\subsection{Why This Matters: The Power of Custom Probabilistic Models}

This example illustrates what makes probabilistic programming fundamentally different from using
off-the-shelf forecasting packages. In a traditional tool, you are limited to the models the
developers anticipated. If the standard TSB model doesn't handle availability constraints---and
no standard implementation does---you are stuck. Your only options are to ignore the problem
(and accept biased forecasts) or to abandon the tool entirely and code a custom solution from scratch.

In NumPyro, the modification required just a few lines of code. We changed how the probability
updates when a zero is observed, conditioning on availability. The rest of the model---the priors,
the likelihood, the inference procedure---remained unchanged. This is the essence of composable
probabilistic programming: you specify the generative process, and the framework handles inference.

The business implications extend beyond forecast accuracy. Because the model explicitly separates
demand from availability, we can now perform \emph{scenario planning}. What would demand look like
if we maintained full availability? Simply set the future availability mask to 1 and generate
forecasts. This capability is invaluable for inventory planning and capacity decisions.

The approach also scales efficiently. Using stochastic variational inference, we can fit this
model to 1,000 time series with 60 observations each in approximately 10 seconds---fast enough
for operational use in a real retail environment.

\begin{figure}[H]
    \centering
    \includegraphics[width=\textwidth]{images/availability_tsb_38_0.png}
    \caption{Cross-validation results demonstrating the model's ability to adapt forecasts
        based on availability information across multiple series.}
    \label{fig:avail_cv}
\end{figure}

While the availability-constrained TSB model handles zeros during stockouts, we also need
tools to handle situations where demand is \emph{partially} met but still higher than recorded
sales---a problem known as demand censoring.

% =============================================================================
% SECTION 10: CENSORED DEMAND
% =============================================================================
\section{Censored Demand Forecasting}

Sales data lies. When a product is out of stock, you observe sales of zero---but true demand may have
been substantial. When inventory is limited to 100 units and you sell all 100, you observe 100 sales---but
true demand may have been 150. In both cases, the observed data \emph{systematically understates}
actual demand. Classical forecasting methods, which take observed values at face value, will
systematically underestimate future demand.

This problem is pervasive in retail and supply chain contexts. Stockouts are common, especially for
popular products and during demand spikes. Inventory constraints limit what can be sold regardless of
underlying demand. Promotional periods may exhaust stock quickly. In all these cases, the historical
data record is \emph{censored}---we observe a lower bound on demand, not demand itself.

\subsection*{Why Probabilistic Methods Excel Here}

Classical forecasting methods have no mechanism for handling censoring. They treat 100 observed sales
as evidence that demand was 100. A probabilistic framework, by contrast, can model the \emph{data-generating
process} explicitly. We don't model ``sales'' directly; we model ``demand'' and then account for the
censoring mechanism that transforms demand into observed sales.

This is a profound conceptual shift. Instead of asking ``what is my forecast given the observed data?''
we ask ``what demand process could have generated the observed data, accounting for the fact that
demand above the inventory level gets recorded as exactly the inventory level?'' The inference
machinery then finds demand parameters consistent with this generative story.

\begin{figure}[H]
    \centering
    \includegraphics[width=\textwidth]{images/demand_8_0.png}
    \caption{Demand data with stockout periods (shaded). During stockouts, observed sales represent
        a lower bound on true demand, not demand itself.}
    \label{fig:censored_data}
\end{figure}

\subsection{Censored Likelihood}

The censored normal likelihood accounts for observations that are right-censored:

\begin{lstlisting}[caption={Censored normal likelihood implementation}]
def censored_normal(loc, scale, y, censored):
    distribution = dist.Normal(loc=loc, scale=scale)
    ccdf = 1 - distribution.cdf(y)
    numpyro.sample(
        "censored_label",
        dist.Bernoulli(probs=ccdf).mask(censored == 1),
        obs=censored
    )
    return numpyro.sample("pred", distribution.mask(censored != 1))
\end{lstlisting}

For uncensored observations, we use the standard normal likelihood. For censored observations, we
contribute only the survival function $P(Y > y)$ to the likelihood.

\begin{figure}[H]
    \centering
    \includegraphics[width=0.9\textwidth]{images/censoring_5_0.png}
    \caption{Illustration of censoring: when inventory is limited, we observe the inventory
        level but not the true demand that exceeded it.}
    \label{fig:censoring_concept}
\end{figure}

\begin{figure}[H]
    \centering
    \begin{tabular}{cc}
        \includegraphics[width=0.45\textwidth]{images/censoring_12_0.png} &
        \includegraphics[width=0.45\textwidth]{images/censoring_15_0.png}
    \end{tabular}
    \caption{Left: Standard model ignoring censoring underestimates parameters. Right: Censored
        likelihood recovers true parameters, correctly accounting for truncated observations.}
    \label{fig:censoring_comparison}
\end{figure}

\subsection{Censored Time Series Model}

Integrating the censored likelihood into a time series model requires modifying the transition function:

\begin{lstlisting}[caption={AR(2) transition with censored likelihood}]
def transition_fn(carry, t):
    y_prev_1, y_prev_2 = carry
    ar_part = phi_1 * y_prev_1 + phi_2 * y_prev_2
    pred_mean = mu + ar_part + seasonal[t]
    # Censored likelihood
    pred = censored_normal(pred_mean, sigma, y[t], censored[t])
    return (pred, y_prev_1), pred
\end{lstlisting}

\begin{figure}[H]
    \centering
    \includegraphics[width=\textwidth]{images/demand_25_0.png}
    \caption{Standard ARIMA forecast ignoring censoring. The model underestimates demand during
        stockout periods and produces biased forecasts.}
    \label{fig:arima_nocensor}
\end{figure}

\begin{figure}[H]
    \centering
    \includegraphics[width=\textwidth]{images/demand_48_0.png}
    \caption{Censored ARIMA forecast correctly inferring latent demand. The model produces
        unbiased forecasts by properly accounting for the censoring mechanism.}
    \label{fig:arima_censored}
\end{figure}

Correctly inferring latent demand is the first step toward effective revenue management.
Next, we explore how to estimate price sensitivity in a hierarchical framework, where these demand
insights are put into action.

% =============================================================================
% SECTION 11: PRICE ELASTICITIES
% =============================================================================
\section{Hierarchical Price Elasticity Models}

Price elasticity estimation---measuring how demand responds to price changes---is a cornerstone of
revenue management and promotional strategy. Estimating these elasticities at the individual product
(SKU) level is challenging because many products have limited price variation or short histories.
Hierarchical modeling allows us to share information across products, regularizing estimates
for data-poor SKUs while allowing data-rich SKUs to express their unique sensitivities.

\subsection{Controlling for Fixed Effects}

To obtain unbiased estimates of price elasticity, it is crucial to control for factors that
simultaneously influence price and demand. We specify a log-log model with two types of fixed effects:

\begin{equation*}
    \log(d_{i,t}) = \alpha_i + \tau_t + \beta_i \log(p_{i,t}) + \epsilon_{i,t}
\end{equation*}

\begin{itemize}
    \item \textbf{Product Fixed Effects ($\alpha_i$)}: These intercepts account for time-invariant
          characteristics of each product, such as brand equity, quality, and typical pack size.
          By including $\alpha_i$, we ensure that $\beta_i$ captures the effect of \emph{price changes}
          rather than baseline demand differences between expensive and cheap products.
    \item \textbf{Time Fixed Effects ($\tau_t$)}: These account for unobserved shocks that affect
          all products simultaneously, such as macroeconomic shifts, holiday effects, or general
          inflation. Without time fixed effects, we might incorrectly attribute a seasonal demand
          surge to a coincidental price drop.
\end{itemize}

\subsection{Hierarchical Structure}

We place a hierarchical prior on the elasticity parameters $\beta_i$, allowing the model to learn
 the category-level distribution:
\begin{align*}
    \beta_i      & \sim \text{Normal}(\mu_\beta, \sigma_\beta) \\
    \mu_\beta    & \sim \text{Normal}(-1, 0.5)                 \\
    \sigma_\beta & \sim \text{HalfNormal}(0.5)
\end{align*}

This structure ensures that for products with little price variation, the estimate shrinks toward the
category mean $\mu_\beta$. For products with rich data, the posterior for $\beta_i$ will be driven
primarily by the product's own price-demand relationship.

\begin{figure}[H]
    \centering
    \includegraphics[width=\textwidth]{images/elasticities_58_0.png}
    \caption{Posterior distributions of price elasticities across SKUs. The hierarchical
        structure effectively handles the diversity of products while providing stable estimates
        across the entire category.}
    \label{fig:elasticities}
\end{figure}

Estimating elasticities from observational data often requires additional constraints
to ensure results remain within physically or economically plausible bounds. We now introduce
a general technique for injecting such domain expertise directly into the inference process.

% =============================================================================
% SECTION 12: MODEL CALIBRATION (SHOWCASE)
% =============================================================================
\section{Model Calibration with Additional Likelihoods}

What if you have domain expertise that the historical data doesn't fully capture? Perhaps you know
from physics that a certain relationship should be approximately linear in a particular regime,
even though your data is too sparse in that region to reveal the pattern. Perhaps you have
experimental results---a randomized controlled trial, an A/B test, a lift test---that provide
ground truth about a causal relationship that observational data alone cannot identify.

Traditional forecasting methods have no mechanism for incorporating such knowledge. You can
adjust priors in a Bayesian model, but priors influence the entire parameter space uniformly.
What if you need to constrain the model only in specific regimes, at specific time points,
or for specific relationships?

This section showcases a powerful technique that addresses exactly this need: treating domain
knowledge as additional likelihood terms. The insight, inspired by the Pyro DLM tutorial
\cite{pyro_dlm_tutorial}, is that in probabilistic programming, there is no fundamental
distinction between ``data we observed'' and ``constraints we want to impose.'' Both are
expressed as likelihood contributions that pull the posterior toward values consistent with
our information.

\subsection{The Core Insight: Priors as Likelihoods}

The technique is elegant in its simplicity. To constrain a latent variable to be approximately
equal to some known value at specific points, we add an ``observation'' of that latent variable
with appropriate uncertainty. The inference machinery then balances this constraint against the
actual data, finding parameter values consistent with both.

This approach is remarkably flexible. We can calibrate coefficients at specific time points
when we have experimental data from those periods. We can constrain model behavior in specific
regimes where we have domain knowledge but sparse data. And as we will demonstrate, we can
calibrate \emph{any} latent component---including flexible nonparametric components like
Gaussian Processes.

\subsection{Key Application: HSGP Calibration for Electricity Demand}

Consider the problem of forecasting electricity demand as a function of temperature
\cite{orduz_electricity_priors}. The relationship is clearly non-linear: at mild temperatures,
demand is low (no heating or cooling needed); as temperature rises, air conditioning loads
increase demand; at very high temperatures, this effect may saturate as most air conditioners
reach maximum capacity.

We model this non-linear relationship using a Hilbert Space Gaussian Process (HSGP)---an efficient
approximation to a full Gaussian Process that scales well with data size. The model also includes
hour-of-day and day-of-week seasonal effects, heteroscedastic noise (demand variance increases
with temperature), and a Student-t likelihood for robustness to outliers.

\begin{figure}[H]
    \centering
    \includegraphics[width=\textwidth]{images/electricity_forecast_6_0.png}
    \caption{Electricity demand data showing the non-linear relationship with temperature.
        At extreme high temperatures, the relationship appears to flatten, but data is sparse
        in this regime, making the pattern difficult to estimate from observations alone.}
    \label{fig:electricity_data}
\end{figure}

The challenge arises at extreme temperatures above 32°C. Historical data in this regime is sparse---heat
waves are rare events. Yet this is precisely where accurate forecasts matter most, because heat waves
strain the electrical grid and accurate demand prediction is critical for preventing blackouts.

Domain knowledge suggests the temperature effect should stabilize around 0.13 in this high-temperature
regime: once most air conditioners are running at full capacity, additional temperature increases
have diminishing marginal effect on demand. But the uncalibrated model, lacking sufficient data,
produces uncertain and oscillating estimates in this region.

The solution is to inject this domain knowledge as a calibration constraint:

\begin{lstlisting}[caption={HSGP calibration for electricity demand}]
# Temperature effect as HSGP (Matern 5/2 approximation)
beta_temperature = numpyro.deterministic(
    "beta_temperature",
    hsgp_matern(x=temperature, nu=5/2, alpha=alpha,
                length=length_scale, ell=ell, m=m)
)

# Calibrate GP for high temperatures (>32C)
temperature_prior_idx = jnp.where(temperature > 32.0)[0]
if temperature_prior_idx is not None:
    numpyro.sample(
        "temperature_prior",
        dist.Normal(loc=0.13, scale=0.01),  # Domain knowledge
        obs=beta_temperature[temperature_prior_idx]
    )
\end{lstlisting}

\begin{figure}[H]
    \centering
    \includegraphics[width=\textwidth]{images/electricity_forecast_46_0.png}
    \caption{Uncalibrated temperature effect from the baseline model. The effect oscillates
        between 0.11 and 0.15 in the high-temperature regime due to data sparsity.}
    \label{fig:electricity_uncalibrated}
\end{figure}

\begin{figure}[H]
    \centering
    \includegraphics[width=\textwidth]{images/electricity_forecast_with_priors_32_0.png}
    \caption{Calibrated temperature effect. The domain knowledge constraint stabilizes
        the effect around 0.13 for temperatures above 32°C while maintaining flexibility
        elsewhere. Forecast metrics (CRPS) remain essentially unchanged.}
    \label{fig:electricity_calibrated}
\end{figure}

\subsection{Why HSGP Calibration is Novel}

Most calibration examples in the literature focus on simple linear coefficients. What makes this
example particularly powerful is that it demonstrates calibration of a \emph{flexible nonparametric
component}. The Gaussian Process can learn arbitrary smooth functions of temperature, yet we can
still inject domain knowledge at specific regimes where we have expertise but sparse data.

This combination---flexibility where data is abundant, constraint where data is sparse---is exactly
what practitioners need. The model retains the ability to discover unexpected nonlinear relationships
in temperature ranges with good data coverage. But in the extreme high-temperature regime, where data
is scarce and forecasts are most critical, the domain knowledge provides stability.

Point estimation methods have no natural analog to this capability. You cannot ``inject knowledge at
specific regimes'' into a maximum likelihood estimate or a gradient-boosted tree. The probabilistic
framework's explicit representation of uncertainty and the generative modeling perspective make this
calibration approach possible.

\subsection{Application: MMM Calibration with Lift Tests}

The same calibration principle applies to Marketing Mix Models (MMM), where the challenge is even
more acute. MMM attempts to estimate the causal effect of advertising spend on sales from observational
data. But observational data is confounded: companies spend more on advertising during periods when
they expect high sales anyway. Lift tests---randomized experiments that measure the true causal effect
of advertising---provide ground truth that can break this confounding.

By incorporating lift test results as calibration likelihoods, we constrain the model's media
effectiveness estimates to be consistent with experimental evidence \cite{pymc_marketing_lift}.
The model learns from both the observational data (which provides information about trends, seasonality,
and relative effects) and the experimental data (which provides unconfounded estimates of absolute effects):

\begin{figure}[H]
    \centering
    \includegraphics[width=0.8\textwidth]{images/pymc-marketing-lift-1.png}
    \caption{MMM calibration with lift tests. Experimental results constrain media effectiveness
        estimates, addressing the unobserved confounders problem common in marketing attribution.}
    \label{fig:mmm_lift}
\end{figure}

The models discussed so far rely on hand-engineered features and explicit state-space
structures. In our final showcase, we explore how to integrate the representative power of
neural networks with the interpretability of state-space models to capture complex, non-linear
patterns in high-dimensional datasets.

% =============================================================================
% SECTION 13: HYBRID DEEP STATE-SPACE MODELS
% =============================================================================
\section{Hybrid Deep State-Space Models}
\label{sec:hybrid}

The final advancement combines neural network components with probabilistic state-space models,
enabling learning of complex patterns while maintaining interpretable structure \cite{orduz_hierarchical_3}.

\subsection{Neural Network Components}

We define embedding and transition networks using Flax NNX:

\begin{lstlisting}[caption={Neural network components for hybrid model}]
class StationEmbedding(nnx.Module):
    def __init__(self, n_stations: int, embedding_dim: int, rngs: nnx.Rngs):
        self.embedding = nnx.Embed(
            num_embeddings=n_stations,
            features=embedding_dim,
            rngs=rngs
        )

    def __call__(self, station_idx):
        return self.embedding(station_idx)


class TransitionNetwork(nnx.Module):
    def __init__(self, embedding_dim: int, hidden_dim: int, rngs: nnx.Rngs):
        self.dense1 = nnx.Linear(embedding_dim + 1, hidden_dim, rngs=rngs)
        self.dense2 = nnx.Linear(hidden_dim, 1, rngs=rngs)

    def __call__(self, embedding, level):
        x = jnp.concatenate([embedding, level[..., None]], axis=-1)
        x = nnx.relu(self.dense1(x))
        return self.dense2(x).squeeze(-1)
\end{lstlisting}

\subsection{Hybrid Model Structure}

The hybrid model augments a local level state-space model with neural network corrections:

\begin{lstlisting}[caption={Hybrid transition function with NN component}]
def transition_fn(carry, t):
    level_prev = carry

    # Station embedding
    embedding = station_embedding(station_idx)

    # NN correction to drift
    nn_correction = transition_network(embedding, level_prev)

    # Local level update with NN correction
    level = level_prev + drift + nn_correction
    level = jnp.where(
        t < t_max,
        level_smoothing * y[:, t] + (1 - level_smoothing) * level,
        level
    )

    mu = level
    pred = numpyro.sample("pred", dist.Normal(loc=mu, scale=noise))

    return level, pred
\end{lstlisting}

The neural network learns station-specific adjustments to the level dynamics that cannot be
captured by the simple local level model alone.

\subsection{SVI Training}

Training hybrid models requires SVI due to the non-conjugate neural network parameters:

\begin{lstlisting}[caption={SVI training for hybrid model}]
# Initialize neural network
nn_params = nnx.state(transition_network)

# AutoNormal guide for probabilistic parameters
guide = AutoNormal(hybrid_model, init_loc_fn=init_to_mean)

# Optax optimizer with OneCycle schedule
scheduler = optax.linear_onecycle_schedule(
    transition_steps=20000,
    peak_value=0.005
)
optimizer = optax.adam(learning_rate=scheduler)

svi = SVI(hybrid_model, guide, optimizer, Trace_ELBO())
svi_result = svi.run(rng_key, 20000, y=data, station_idx=stations)
\end{lstlisting}

\subsection{When Neural Networks Help}

Neural network corrections are most valuable when:
\begin{itemize}
    \item Multiple related time series share complex non-linear patterns
    \item Standard state-space models leave systematic residual structure
    \item Sufficient data exists to learn embeddings without overfitting
\end{itemize}

For simpler problems, pure state-space models often suffice and provide better interpretability.

% =============================================================================
% SECTION 14: CONCLUSION
% =============================================================================
\section{Conclusion}

\subsection{The Paradigm Shift}

Throughout this article, we have explored a fundamental shift in how we approach time series forecasting.
Traditional methods ask: ``Given historical data, what is my best guess for future values?'' Probabilistic
forecasting asks a different question: ``What is the full distribution of possible future values, and
how confident should I be in different outcomes?''

This shift from point estimates to distributions has profound implications for decision-making.
A retailer deciding on inventory levels needs to know not just the expected demand, but the
probability of demand exceeding various thresholds. An energy grid operator planning for peak
load needs to understand the range of possible demand scenarios, not just the central forecast.
A marketing team allocating budget needs to quantify uncertainty in media effectiveness estimates,
not just obtain a single ``best'' number.

But probabilistic forecasting offers more than uncertainty quantification. As we have demonstrated
throughout this article, probabilistic programming enables us to directly incorporate business
constraints, domain knowledge, and potentially causal relationships into our models. When sales
data is censored by stockouts, we model the censoring mechanism. When different regions share
common patterns, we let them borrow statistical strength through hierarchical structures. When
we have experimental evidence about causal effects, we inject it as calibration constraints.
This flexibility to encode domain expertise directly into model structure is what makes
probabilistic forecasting truly transformative for business applications.

\subsection{Matching Problems to Solutions}

A key theme of this article has been matching the right technique to the right problem.
Not every forecasting challenge requires sophisticated methods---sometimes a simple exponential
smoothing model suffices. The value of understanding the full toolkit is knowing when each
approach provides genuine advantage.

\textbf{When data is sparse across related series}, hierarchical models shine. The tourism
forecasting example demonstrated how regions with limited history can borrow strength from
data-rich regions. Classical methods, which treat each series independently, cannot achieve
this information sharing.

\textbf{When stockouts mask true demand}, censored likelihoods are essential. Standard methods
treat observed sales as ground truth, systematically underestimating demand. Only by modeling
the data-generating process---including the censoring mechanism---can we recover unbiased
demand estimates.

\textbf{When zeros arise from availability constraints rather than lack of demand}, the
availability-constrained TSB model provides the solution. By modifying a single line in the
transition function, we encode the distinction between true zero demand and supply-constrained
zeros directly into the model structure.

\textbf{When domain expertise exceeds what historical data can reveal}, calibration via
additional likelihoods enables injecting that knowledge. The HSGP calibration example showed
how to constrain a flexible Gaussian Process in specific regimes where we have expertise but
sparse data.

\textbf{When scale matters}---thousands of time series to forecast---stochastic variational
inference with GPU acceleration makes probabilistic methods practical. What would take hours
with MCMC can run in minutes with SVI, enabling rapid iteration and operational deployment.

\subsection{Limitations and When to Use Simpler Methods}

Intellectual honesty requires acknowledging when probabilistic methods are \emph{not} the right
choice. Strong baseline models are often hard to beat, and the additional complexity of
probabilistic programming may not be justified for every forecasting task.

\textbf{Computational cost}: Probabilistic methods are more computationally expensive than
simple exponential smoothing or ARIMA models. If you need to forecast millions of time series
and point forecasts suffice, simpler methods may be more practical.

\textbf{Expertise required}: Specifying a probabilistic model requires understanding of
statistical modeling, prior selection, and inference diagnostics. Teams without this expertise
may be better served by automated tools like AutoETS or Prophet that make reasonable default
choices.

\textbf{When baselines suffice}: For stable, well-behaved time series with abundant historical
data and no special constraints, classical methods often perform comparably. The advantages of
probabilistic forecasting emerge most clearly when data has quirks (censoring, intermittency,
hierarchical structure) or when domain knowledge needs to be incorporated.

\textbf{Model misspecification risk}: A misspecified probabilistic model can produce confidently
wrong forecasts with misleadingly narrow uncertainty bands. Proper model checking, posterior
predictive checks, and out-of-sample validation are essential but require additional effort.

The pragmatic recommendation is to start simple, establish strong baselines, and add complexity
only when it demonstrably improves performance on held-out data or enables incorporating domain
knowledge that simpler methods cannot capture.

\subsection{Future Directions}

The field of probabilistic forecasting continues to evolve rapidly. Several directions hold
particular promise:

\textbf{State space models in PyMC-Extras}: Recent development of structural time series
components in PyMC provides composable building blocks (local linear trend, seasonal components,
regression effects) with Kalman filter-based inference.

\textbf{Foundation models for time series}: Large pre-trained models like Chronos and TimeGPT
offer zero-shot forecasting capabilities. How to combine these with domain-specific probabilistic
structure remains an open question.

\textbf{Causal extensions}: The calibration technique we demonstrated with lift tests is a step
toward causal forecasting. Methods like Causal DeepAR extend these ideas to deep learning
architectures, enabling counterfactual prediction.

\textbf{Integration with decision optimization}: The natural next step after probabilistic
forecasting is decision-making under uncertainty. Connecting forecast distributions to
optimization objectives (inventory costs, energy dispatch, marketing ROI) closes the loop
from prediction to action.

\subsection{Broader Impact}

The techniques presented in this article enable a different kind of relationship between
forecasters and decision-makers. Instead of delivering a single number that stakeholders
must trust or reject, probabilistic forecasting delivers a range of scenarios with associated
probabilities. This transparency enables more informed decision-making and more honest
communication about uncertainty.

Perhaps more importantly, the ability to encode domain knowledge directly into models
changes the role of subject matter experts. In traditional forecasting, domain knowledge
enters informally---an analyst might ``adjust'' a model's output based on intuition.
In probabilistic programming, domain knowledge enters formally---as priors, constraints,
or structural assumptions that become part of the model itself. This formalization makes
assumptions explicit, auditable, and updatable as understanding evolves.

The combination of rigorous uncertainty quantification, flexible model specification,
and the ability to incorporate domain expertise makes probabilistic forecasting a powerful
tool for any organization serious about data-driven decision-making under uncertainty.

\subsection{Acknowledgments}

We gratefully acknowledge the NumPyro core developers---Du Phan, Neeraj Pradhan, and Martin
Jankowiak---for creating and maintaining this excellent library. The Pyro team at Uber AI Labs,
including Eli Bingham, Fritz Obermeyer, and Noah Goodman, laid the foundation with Pyro
\cite{bingham2019pyro}. The JAX team at Google enables the performance that makes these
methods practical.

We also thank the broader probabilistic programming community, whose contributions through
blog posts, tutorials, and open-source code have made these techniques accessible to
practitioners. Special thanks to the PyMC Labs team for their work on probabilistic
forecasting for business applications, which inspired many of the examples in this article.

\bibliographystyle{acm}
\bibliography{references}

\end{document}
